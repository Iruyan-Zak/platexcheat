\documentclass[dvipdfmx,11pt]{jarticle}
\usepackage{otf,hyperref}
\title{p\LaTeXe をもっと便利に}
\author{朝倉卓人\thanks{\texttt{wtsnjp@gmail.com}}}

\begin{document}
\maketitle

\section{発展的なp\LaTeXe の利用法}
\subsection{マクロ}
本シートで紹介しているのはp\LaTeXe 標準の命令や環境
ばかりだがp\LaTeXe にはユーザ独自の\emph{マクロ}を
定義するしくみがある.

\subsection{p\LaTeXe 関連の情報や成果物を入手する}
近年人気の\TeX ディストリビューション(\TeX\ Liveなど)
を利用している場合は,自力で入手しなくとも数多くのクラスや
パッケージ最初からインストールされている.より多くの情報や
成果物が必要な場合は表\ref{表:サイト集}のサイトを訪れる.

\begin{table}[bth!]
\centering
\caption{\emph{便利なサイト}.英語のサイトも含む.}
\label{表:サイト集}
\begin{tabular}{ll}
\emph{サイト名} & \emph{説明} \\ \hline
\href{https://www.ctan.org/}{CTAN} 
  & 世界から\TeX 関連成果物が集められている. \\
\href{https://texwiki.texjp.org/}{{\TeX} Wiki}
  & \TeX に関する日本語情報が集積. \\
\end{tabular}
\end{table}
\end{document}