% typeset: ptex2pdf -l latexsheet.tex
\documentclass[10pt,a4paper,landscape,draft]{jarticle}

% 使用パッケージ
\usepackage[small]{titlesec}
\usepackage{geometry}
\usepackage{multicol}
\usepackage{shortvrb}
\usepackage{plistings}
\usepackage{siunitx}
\usepackage{calc}     % いずれ除去
\usepackage{otf}
\usepackage{url}

% ドキュメント設定
%% ページ
\geometry{top=1cm, left=1cm, right=1cm, bottom=1cm}
\pagestyle{empty}

%% 見出し
\setcounter{secnumdepth}{0}
\titlespacing{\section}{0pt}{5pt}{0pt}
\titlespacing{\subsection}{0pt}{0pt}{0pt}
\titlespacing{\subsubsection}{0pt}{0pt}{0pt}

%% 寸法
\setlength{\parindent}{0pt}
\setlength{\parskip}{0pt plus 0.5ex}

%% コード
\MakeShortVerb{\|}
\lstset{
  language={[LaTeX]TeX},
  basicstyle={\small\ttfamily},
  commentstyle={\small\itshape},
  keywordstyle={\small\ttfamily\bfseries},
  columns=[l]{fullflexible}}

% マクロ定義
%% 列挙用テーブル
\makeatletter
\def\set@etsep#1#2{\def\etcolsep{#1}\def\etitemsep{#2}}
\newenvironment{entable}[3][\quad\qquad]{%
  \set@etsep#1\relax\relax
  \begin{tabular}{%
    @{}*{\the\numexpr#3-1}{*{\the\numexpr#2-1}{l@{\etcolsep}}l@{\etitemsep}}%
    *{\the\numexpr#2-1}{l@{\etcolsep}}l@{}}}{%
  \end{tabular}}
\newcommand{\mergecol}[2]{\multicolumn{#1}{@{}l@{}}{#2}}
\makeatother

%% ユーザ用マクロ
\chardef\vbar=`\| \chardef\bs=`\\
\newcommand{\mboxtt}[1]{\mbox{\texttt{#1}}}
\newcommand{\Meta}[1]{$\langle$\mbox{}\emph{#1}\mbox{}$\rangle$}
\newcommand{\psp}{\phantom{ }}                                       % いずれ除去
\newlength{\MyLen}                                                   % いずれ除去
\newcommand{\BibTeX}{\textnormal{\textsc{Bib}\TeX}}

% -----------------------------------------------------------------------

\begin{document}

% 三段組を開始
\begin{multicols}{3}
\setlength{\premulticols}{1pt}
\setlength{\postmulticols}{1pt}
\setlength{\multicolsep}{1pt}
\setlength{\columnsep}{2pt}

% 文字サイズ
\footnotesize

% タイトル
\begin{center}
\Large
\textbf{p\LaTeXe チートシート}
\end{center}

\section{p\LaTeXe の基本}

\begin{entable}{2}{1}
制御綴 & `\texttt{\bs}' で始まる命令.引数は |{}|,省略可能な引数は |[]| で囲う \\
環境 & |\begin{|\Meta{環境名}|}| と |\end{|\Meta{環境名}|}| に挟まれた部分 \\
コメント & `|%|' 以降は行末までコメントとして扱われる(無視される) \\
\end{entable}

すべてではないが,多くの命令や環境は入れ子にできる.

\subsection{長さの単位}

\begin{entable}{2}{2}
|cm|/|mm|/|in|
  & \mergecol{3}{一般的な長さの単位($\SI{1}{in}=\SI{25.4}{mm}$)} \\
|pt|/|bp|/|sp|
  & \mergecol{3}{ポイント($\SI{72.27}{pt}=\SI{72}{bp}=\SI{65536}{sp}=\SI{1}{in}$)} \\
|Q|/|H|
  & \mergecol{3}{級と歯($\SI{1}{Q}=\SI{1}{H}=\SI{0.25}{mm}$)} \\
|em|/|ex|\phantom{/|pt|} & |M| の幅と |x| の高さ & |zw|/|zh| & 和文文字の幅と高さ \\
\end{entable}

\section{プリアンブル}

文書はクラス宣言で開始:
|\documentclass[|\Meta{オプション}|]{|\Meta{クラス}|}|.\break
本文はdocument環境に記述.|\documentclass| と |\begin{document}| の間は
\emph{プリアンブル}と呼ばれ,文書全体に適用する設定などを記述する.

\subsection{標準クラス}

\begin{entable}{2}{2}
|jbook| & 横書き書籍 & |tbook| & 縦書き書籍 \\
|jarticle| & 横書き論文 & |tarticle| & 縦書き論文 \\
|jreport| & 横書きレポート & |treport| & 縦書きレポート \\
\end{entable}

クラスによってデフォルトのオプションや使える見出し命令などが異なる.

\subsection{主な標準クラスオプション}

\begin{entable}[\enspace\qquad]{2}{2}
\mergecol{4}{%
  |a4paper|/|a5paper|/|b4paper|/|b5paper|\enspace 用紙サイズ(B系列はJIS規格)} \\
|a4j|/|a5j|/|b4j|/|b5j| & 上記より余白が狭い & |landscape| & 用紙を横向き \\
|10pt|/|11pt|/|12pt| & 文字サイズ & |twocolumn| & 二段組 \\
|titlepage| & タイトルを独立ページ & |twoside| & 両面印刷 \\
|leqno| & 数式番号を左側に表示 & |dvipdfmx| & dviドライバ \\
|fleqn| & 別行立て数式を左寄せ & |tombo| & トンボを表示 \\
\end{entable}

クラスオプションに指定したものは読み込んだパッケージへも適用される.

\subsection{文書情報}

\begin{entable}[\enspace\quad]{2}{2}
|\author{|\Meta{著者}|}|
  & \mergecol{3}{文書の著者.複数のときは \texttt{\bs and} で区切る} \\
|\thanks{|\Meta{注釈}|}|
  & \mergecol{3}{著者名に脚注(主に所属機関)を付ける} \\
|\title{|\Meta{タイトル}|}| & 文書のタイトル & |\date{|\Meta{日付}|}| & 文書の作成日 \\
\end{entable}

本文中 |\maketitle| を記述すると上記の情報を基に表題が出力される.

\subsection{ページスタイル}

\begin{entable}[\enspace\qquad]{2}{2}
|empty| & 空のヘッダ・フッタ & |plain| & ノンブル(ページ番号) \\
|headings| & 柱とノンブル & |myheadings| & カスタムの柱
\end{entable}

|\pagestyle{|\Meta{スタイル}|}| の形式で指定.この命令は本文中でも利用可能.
|\thispagestyle{|\Meta{スタイル}|}| により一時的な変更もできる.|myheadings| 指定時
は |\markboth{|\Meta{偶数ページ用}|}{|\Meta{奇数ページ用}|}| で柱を設定.

\subsection{各種設定など}

\begin{entable}{2}{1}
|\usepackage[|\Meta{オプション}|]{|\Meta{パッケージ}|}| & パッケージ読み込み \\
|\setcounter{|\Meta{カウンタ}|}{|\Meta{数値}|}| & カウンタの値を設定 \\
|\setlength{|\Meta{寸法を表す制御綴}|}{|\Meta{長さ}|}| & 寸法を設定 \\
|\newcommand{|\Meta{制御綴}|}[|\Meta{引数の数}|]{|\Meta{定義}|}| & マクロの定義 \\
%|\newenvironment{|\Meta{環境名}|}{|\Meta{開始}|}{|\Meta{終了}|}| & 環境の定義 \\
\end{entable}

|\usepackage| はプリアンブルのみで利用可.他の命令は中級者以上向け.

\section{文書の論理構造}

%\subsection{見出し}

\begin{entable}[\enspace\quad]{2}{3}
|\part| & 部 & |\chapter| & 章 & |\section| & 節 \\
|\subsection| & 小節 & |\subsubsection| & 小々節 & |\pragraph| & 段落 \\
|\subpragraph| & 小段落 \\
\end{entable}

これらの命令は共通の書式をもつ:\Meta{制御綴}|[|\Meta{目次用の見出し}|]{|\Meta{見出し}|}|.
\Meta{制御綴}の直後に |*| を付けた場合,見出し番号が付かず目次にも載らない.

\subsection{箇条書き}

\begin{entable}[\enspace\qquad]{2}{2}
|\begin{itemize}| & 記号付き & |\begin{enumerate}| & 連番付き \\
|\begin{description}| & 見出し付き \\
\end{entable}

各項目は上記の環境中で |\item |\Meta{テキスト} で表現.|\item[|\Meta{ラベル}|]|\break
のようにすると当該項目のラベルのみを変更(description環境では必須).

\subsection{フロート}

\begin{entable}{2}{2}
|\begin{figure}[|\Meta{位置}|]| & 図 & |\begin{table}[|\Meta{位置}|]| & 表 \\
\end{entable}

各環境中 |\caption[|\Meta{図表目次用}|]{|\Meta{説明文}|}| でキャプションを付けられる.
\Meta{位置}は次の文字の組み合わせで指定:$\mboxtt{t}=\mbox{ページ上部}$,
$\mboxtt{b}=\mbox{ページ下部}$,$\mboxtt{p}=\mbox{独立ページ}$,$\mboxtt{h}=\mbox{その場}$,
$\mboxtt{!}=\mbox{条件が悪くても指定位置に配置}$.

\subsection{引用と注釈}

\begin{entable}[\enspace\qquad]{2}{2}
|\begin{quote}| & 短い引用 & |\begin{quotation}| & 複数段落の引用 \\
\mergecol{2}{\texttt{\bs footnote[}\Meta{番号}\texttt{]\{}\Meta{脚注}\texttt{\}}}
  & \mergecol{2}{\texttt{\bs marginpar\{}\Meta{傍注}\texttt{\}}} \\
\end{entable}

%|\footnote| が使えない場面もある.その場合 |\footnotemark[|\Meta{番号}|]| で
%記号を付け,後から |\footnotetext[|\Meta{番号}|]{|\Meta{脚注}|}| でテキストを与える.

\section{テキストプロパティ}

\subsection{書体}

\begin{entable}[\enspace]{3}{1}
|\textmc{|\Meta{テキスト}|}| & |{\mcfamily |\Meta{テキスト}|}| & \textmc{明朝体} \\
|\textgt{|\Meta{テキスト}|}| & |{\gtfamily |\Meta{テキスト}|}| & \textgt{ゴシック体} \\
|\textrm{|\Meta{テキスト}|}| & |{\rmfamily |\Meta{テキスト}|}| & \textrm{Roman family} \\
|\textsf{|\Meta{テキスト}|}| & |{\sffamily |\Meta{テキスト}|}| & \textsf{Sans serif family} \\
|\texttt{|\Meta{テキスト}|}| & |{\ttfamily |\Meta{テキスト}|}| & \texttt{Typewriter family} \\
|\textmd{|\Meta{テキスト}|}| & |{\mdseries |\Meta{テキスト}|}| & \textmd{Medium series} \\
|\textbf{|\Meta{テキスト}|}| & |{\bfseries |\Meta{テキスト}|}| & \textbf{Bold series} \\
|\textup{|\Meta{テキスト}|}| & |{\upshape |\Meta{テキスト}|}| & \textup{Upright shape} \\
|\textit{|\Meta{テキスト}|}| & |{\itshape |\Meta{テキスト}|}| & \textit{Italic shape} \\
|\textsl{|\Meta{テキスト}|}| & |{\slshape |\Meta{テキスト}|}| & \textsl{Slanted shape} \\
|\textsc{|\Meta{テキスト}|}| & |{\scshape |\Meta{テキスト}|}| & \textsc{Small Caps shape} \\
\end{entable}

|\textnormal{|\Meta{テキスト}|}|, |\normalfont| を用いるとファミリ・シリーズ・シェイプを
同時にデフォルト設定へ戻すことができる.

\subsection{装飾}

\begin{entable}[\enspace]{3}{1}
|\emph{|\Meta{テキスト}|}| & |{\em |\Meta{テキスト}|}| & \emph{強調}(効果は環境依存) \\
|\underline{|\Meta{テキスト}|}| & & \underline{下線} \\
\end{entable}

\subsection{フォントサイズ}

\bgroup
\setlength{\columnsep}{14pt}
\begin{multicols}{2}
\begin{tabbing}
|\footnotesize| \= \kill
|\tiny| \> \tiny{極小}\\
|\scriptsize| \> \scriptsize{スクリプトサイズ}\\
|\footnotesize| \> \footnotesize{脚注サイズ}\\
|\small| \> \small{小さい}\\
|\normalsize| \> \normalsize{標準}\\
|\large| \> \large{大きい}\\
|\Large| \= \Large{超大きい}\\
|\LARGE| \> \LARGE{超々大きい}\\
|\huge| \> \huge{巨大}\\
|\Huge| \> \Huge{超巨大}
\end{tabbing}
\end{multicols}
\egroup

これらの宣言は \verb!{\small! \ldots\verb!}! のような形で利用するか,ブレースなし
で以降の文書全体のフォントサイズを変更する.

\subsection{入力通りに出力}

\begin{entable}[\enspace]{2}{1}
|\verb!|\Meta{テキスト}|!| & デリミタ(ここでは `\texttt{!}')間の内容をそのまま出力 \\
|\begin{verbatim}| & 入力通りに出力される環境 \\
\end{entable}

|\verb*| や |\begin{verbatim*}| を用いると空白が `\verb*! !' で明示される.

\subsection{配置}

\begin{entable}[\qquad]{3}{1}
|\begin{center}| & |\centering| & 中央寄せ \\
|\begin{flushleft}| & |\raggedright| & 左寄せ \\
|\begin{flushright}| & |\raggedleft| & 右寄せ \\
\end{entable}

\section{空白・行・ページ}

\subsection{水平スペース(文字送り方向)}

\begin{entable}[\enspace\quad]{2}{4}
\verb*!\ ! & 空白 & |\enspace| & \SI{0.5}{em} & |\quad| & \SI{1}{em} & |\,| & 小さな空白 \\
|~| & 改行禁止 & \mergecol{2}{\texttt{\bs hspace\{}\Meta{長さ}\texttt{\}}}
  & |\qquad| & \SI{2}{em} & |\!| & 負の空白 \\
\end{entable}

|\hfil|, |\hfill| は無限に伸びる空白で後者がより強力.英大文字で終わる文のピリオド直前に
置く |\@| やイタリック補正の |\/| もスペーシングに影響.

\subsection{垂直スペース(行送り方向)}

\begin{entable}[\enspace]{6}{1}
|\smallskip| & |\medskip| & |\bigskip| & |\vspace{|\Meta{長さ}|}| & |\vfil| & |\vfill| \\
\end{entable}

|\phantom{|\Meta{テキスト}|}| を用いると\Meta{テキスト}と同じ寸法の空白を作る.

\subsection{行とページ}

\begin{entable}[\enspace\enspace]{2}{2}
|\par| & 改段落(空行と同じ) & |\noindent| & インデントしない \\
|\\*| & 改行(改ページなし) & |\\[|\Meta{長さ}|]| & 改行(縦スペース付加) \\
|\newpage| & 改ページ(段移動) & |\clearpage| & 改ページ(全図表出力) \\
\end{entable}

|\break| は段落内では改行(|\\| とは挙動が異なる),段落間では改ページ.

\section{記号・その他}

\subsection{一般的な記号}

\begin{entable}[\enspace\qquad]{2}{4}
\& & |\&| & \_ & |\_| & \dag & |\dag| & \textbackslash & |\textbackslash| \\
\$ & |\$| & \P & |\P| & \ddag & |\ddag| & \textbullet & |\textbullet| \\
\% & |\%| & \S & |\S| & \pounds & |\pounds| & \texttrademark & |\texttrademark| \\
\# & |\#| & \dots & |\dots| & \copyright & |\copyright| & \textregistered & |\textregistered| \\
\end{entable}

\subsection{アクセント類}

\begin{entable}[\enspace\qquad]{2}{6}
!` & |!`| & ?` & |?`| & \.a & |\.a| & \'a & |\'a| & \"a & |\"a| & \`a & |\`a| \\
\^a & |\^a| & \=a & |\=a| & \~a & |\~a| & \aa & |\aa| & \AA & |\AA| & \ae & |\ae| \\
\AE & |\AE| & \b a & |\b a| & \c c & |\c c| & \d a & |\d a| & \H a & |\H a| & \i & |\i| \\
\j & |\j| & \l & |\l| & \L & |\L| & \o & |\o| & \O & |\O| & \oe & |\oe| \\
\OE & |\OE| & \r a & |\r a| & \ss & |\ss| & \t oo & |\t oo| & \u a & |\u a| & \v a & |\v a| \\
\end{entable}

\subsection{括弧類}

\begin{entable}{2}{6}
` & |`| & `` & |``| & \{ & |\{| & \lbrack & |[| & ( & |(| & \textless & |\textless| \\
' & |'| & '' & |''| & \} & |\}| & \rbrack & |]| & ) & |)| & \textgreater & |\textgreater| \\
\end{entable}

\subsection{ダッシュ類}

\begin{entable}[\qquad]{4}{1}
ハイフン & |-| & X-ray & 単語内,2単語の連結 \\
en-dash & |--| & 1--5 & 範囲を表す場合など \\
em-dash & |---| & Yes---or no? & 挿入句の導入など \\
\end{entable}

\subsection{その他}

\begin{entable}{2}{1}
|\today| & 現在の日付.|\西暦|, |\和暦|で形式を変更可能 \\
|\TeX|/|\LaTeX|/|\LaTeXe| & \TeX/\LaTeX/\LaTeXe \\
\end{entable}

\section{表組み}

\subsection{tabular環境}

\begin{entable}{1}{2}
|\begin{tabular}{|\Meta{列指定}|}| & |\begin{tabular*}{|\Meta{幅}|}{|\Meta{列指定}|}| \\
\end{entable}

\Meta{列指定}は次の文字を並べて指定:$\mboxtt{l}=\mbox{左寄せの列}$,
$\mboxtt{c}=\mbox{中央寄せの列}$,$\mboxtt{r}=\mbox{右寄せの列}$,
$\mboxtt{p\{\Meta{幅}\}}=\mbox{寸法指定の列}$,
$\mboxtt{@\{\Meta{区切り}\}}=\mbox{列間の区切り}$,
$\mbox{\texttt{\vbar}}=\mbox{表全体の縦罫線}$.
具体的な使用例は本シート末尾の文書サンプル参照.

%\subsubsection{要素}

\begin{entable}{2}{2}
|&| & 列の区切り & |\hline| & 表全体の横罫線 \\
|\\| & 行の終端 & |\cline{|\Meta{開始}|-|\Meta{終了}|}| & 範囲指定の横罫線 \\
\end{entable}

|\multicolumn{|\Meta{列数}|}{|\Meta{列指定}|}{|\Meta{テキスト}|}| を用いると,
\Meta{列数}個の列を結合したセルを作ることができる.

\subsection{tabbing環境}

\begin{entable}{2}{2}
|\=| & タブ位置を設定 & |\>| & 次のタブ位置に移動 \\
|\\| & 行の終端 & |\kill| & 非表示行(タブ位置設定)の終端 \\
\end{entable}

\section{数式モード}
数式モードを使うには,該当箇所を `\verb!$!' で囲うかequation環境を用いる.

\begin{tabular}{@{}l@{\hspace{1em}}l@{\hspace{2em}}l@{\hspace{1em}}l@{}}
上付き$^{x}$            &
\verb!^{x}!             &  
下付き$_{x}$            &
\verb!_{x}!             \\  
$\frac{x}{y}$           &
\verb!\frac{x}{y}!      &  
$\sum_{k=1}^n$          &
\verb!\sum_{k=1}^n!     \\  
$\sqrt[n]{x}$           &
\verb!\sqrt[n]{x}!      &  
$\prod_{k=1}^n$         &
\verb!\prod_{k=1}^n!    \\ 
\end{tabular}

\subsection{記号類}

% The ordering of these symbols is slightly odd.  This is because I had to put all the
% long pieces of text in the same column (the right) for it all to fit properly.
% Otherwise, it wouldn't be possible to fit four columns of symbols here.

\begin{tabular}{@{}l@{\hspace{1ex}}l@{\hspace{1em}}l@{\hspace{1ex}}l@{\hspace{1em}}l@{\hspace{1ex}} l@{\hspace{1em}}l@{\hspace{1ex}}l@{}}
$\leq$          &  \verb!\leq!  &
$\geq$          &  \verb!\geq!  &
$\neq$          &  \verb!\neq!  &
$\approx$       &  \verb!\approx!  \\
$\times$        &  \verb!\times!  &
$\div$          &  \verb!\div!  &
$\pm$           & \verb!\pm!  &
$\cdot$         &  \verb!\cdot!  \\
$^{\circ}$      & \verb!^{\circ}! &
$\circ$         &  \verb!\circ!  &
$\prime$        & \verb!\prime!  &
$\cdots$        &  \verb!\cdots!  \\
$\infty$        & \verb!\infty!  &
$\neg$          & \verb!\neg!  &
$\wedge$        & \verb!\wedge!  &
$\vee$          & \verb!\vee!  \\
$\supset$       & \verb!\supset!  &
$\forall$       & \verb!\forall!  &
$\in$           & \verb!\in!  &
$\rightarrow$   &  \verb!\rightarrow! \\
$\subset$       & \verb!\subset!  &
$\exists$       & \verb!\exists!  &
$\notin$        & \verb!\notin!  &
$\Rightarrow$   &  \verb!\Rightarrow! \\
$\cup$          & \verb!\cup!  &
$\cap$          & \verb!\cap!  &
$\mid$          & \verb!\mid!  &
$\Leftrightarrow$   &  \verb!\Leftrightarrow! \\
$\dot a$        & \verb!\dot a!  &
$\hat a$        & \verb!\hat a!  &
$\bar a$        & \verb!\bar a!  &
$\tilde a$      & \verb!\tilde a!  \\

$\alpha$        &  \verb!\alpha!  &
$\beta$         &  \verb!\beta!  &
$\gamma$        &  \verb!\gamma!  &
$\delta$        &  \verb!\delta!  \\
$\epsilon$      &  \verb!\epsilon!  &
$\zeta$         &  \verb!\zeta!  &
$\eta$          &  \verb!\eta!  &
$\varepsilon$   &  \verb!\varepsilon!  \\
$\theta$        &  \verb!\theta!  &
$\iota$         &  \verb!\iota!  &
$\kappa$        &  \verb!\kappa!  &
$\vartheta$     &  \verb!\vartheta!  \\
$\lambda$       &  \verb!\lambda!  &
$\mu$           &  \verb!\mu!  &
$\nu$           &  \verb!\nu!  &
$\xi$           &  \verb!\xi!  \\
$\pi$           &  \verb!\pi!  &
$\rho$          &  \verb!\rho!  &
$\sigma$        &  \verb!\sigma!  &
$\tau$          &  \verb!\tau!  \\
$\upsilon$      &  \verb!\upsilon!  &
$\phi$          &  \verb!\phi!  &
$\chi$          &  \verb!\chi!  &
$\psi$          &  \verb!\psi!  \\
$\omega$        &  \verb!\omega!  &
$\Gamma$        &  \verb!\Gamma!  &
$\Delta$        &  \verb!\Delta!  &
$\Theta$        &  \verb!\Theta!  \\
$\Lambda$       &  \verb!\Lambda!  &
$\Xi$           &  \verb!\Xi!  &
$\Pi$           &  \verb!\Pi!  &
$\Sigma$        &  \verb!\Sigma!  \\
$\Upsilon$      &  \verb!\Upsilon!  &
$\Phi$          &  \verb!\Phi!  &
$\Psi$          &  \verb!\Psi!  &
$\Omega$        &  \verb!\Omega!  
\end{tabular}

\section{相互参照・目次・索引}
\subsection{相互参照}
\settowidth{\MyLen}{\texttt{.footnote.テキスト..} }
\begin{tabular}{@{}p{\the\MyLen}%
                @{}p{\linewidth-\the\MyLen}@{}}
\verb!\label{!\Meta{マーカ}\verb!}!      &  相互参照用のマーカをセット. \\
\verb!\ref{!\Meta{マーカ}\verb!}!        &  マーカの位置(節番号など)を出力. \\
\verb!\pageref{!\Meta{マーカ}\verb!}!    &  マーカのページ番号を出力. \\
\end{tabular}

\subsection{目次}
\settowidth{\MyLen}{\texttt{.tableofcontents} }
\begin{tabular}{@{}p{\the\MyLen}%
                @{}p{\linewidth-\the\MyLen}@{}}
\verb!\tableofcontents!  & 目次を印刷. \\
\verb!\tableoffigures!   & 図目次を印刷. \\
\verb!\tableoftables!    & 表目次を印刷. \\
\end{tabular}

目次を表示させるためには\texttt{platex}を複数回実行する必要がある.

\section{\LaTeX 文書のサンプル}
\lstinputlisting{platexsample.tex}

\rule{.3\linewidth}{.25pt}
\scriptsize

\LaTeXe 版 {\copyright} 2006 Winston Chang \\
\url{http://www.stdout.org/~winston/latex/} \\
p\LaTeXe 版 {\copyright} 2017 Takuto Asakura \\
\url{https://wtsnjp.com/pdf/platexsheet.pdf}

バージョン:1.1, 作成日:2017/01/03

\end{multicols}
\end{document}
